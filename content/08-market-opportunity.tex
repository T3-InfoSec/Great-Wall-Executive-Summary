% 08-market-opportunity.tex - Total addressable market analysis

\section{Total Addressable Market}

\begin{table}[H]
\centering
\begin{tabularx}{\linewidth}{l Y Y Y}
\toprule
Market Segment & Global TAM & Serviceable (SAM) & Target Share \\\midrule
Subscription Users & \numint{\tamSubsGlobal}\cite{chainalysis2024,triple2023,statista2024pwd,lastpass2024} & \numint{\samSubs} & \num{\targetShareSubs}\% (\num{\targetSubsCountk}k) \\
\bottomrule
\end{tabularx}
\end{table}
\textit{Note: TAM includes password manager users\cite{statista2024pwd,lastpass2024}, private security/insurance customers\cite{alliedmarket2023}, and physical vault users\cite{grandview2024,mordor2024} seeking digital alternatives. Merchandise buyers overlap with primary segments and serve as a community engagement tool and minor, non-projected revenue stream.}

\subsection{Market Share Benchmarks}
The target market shares are based on comparable first-mover and strategic partnership successes:
\begin{itemize}
  \item \textbf{Subscription (\num{\targetShareSubs}\% of SAM):} Aligned with Coinbase's 15\% crypto exchange capture\cite{coinbase2021}, Stripe's 20\% payment processing share\cite{stripe2023}, and LastPass/1Password's 10-15\% password management penetration\cite{lastpass2023}
  \item \textbf{Strategic Advantages:} Partnership with market leader provides distribution channels, brand credibility, and accelerated customer acquisition typically doubling organic growth rates\cite{reforge2022,hagiu2015}
\end{itemize}

\subsection{Market Dynamics}
\begin{itemize}
  \item \textbf{Bitcoin Adoption:} Growing mainstream adoption drives demand for security tools
  \item \textbf{Self-Sovereignty Trend:} "Not your keys, not your coins" philosophy expanding market
  \item \textbf{Privacy Concerns:} Increasing demand for anonymous computation services
  \item \textbf{Underserved Market:} Limited competition in anonymous marketplace segment
  \item \textbf{Adjacent Markets:} TAM includes password manager users seeking stronger security solutions\cite{statista2024pwd,lastpass2024}, customers of private security companies/violence insurance exploring digital alternatives\cite{alliedmarket2023}, and physical vault users transitioning to digital security\cite{grandview2024,mordor2024}
\end{itemize}

\subsection{2025 Market Inflection}
The year 2025 witnessed three seemingly unrelated developments:
\begin{enumerate}
  \item A sharp rise in reported wrench attacks targeting cryptocurrency holders---a trend that continues to the time of writing;
  \item The breaking of Bitcoin's historical 3-up-1-down price cycle, with Bitcoin closing the year at a loss in what should have been a positive year;
  \item A sustained bull market in precious metals.
\end{enumerate}

While the causal links between these events remain unproven, a plausible narrative emerges: both large and small crypto holders, increasingly viewing ownership as a security liability, may be diversifying into physical assets. The reasoning is straightforward: if holding cryptocurrency incurs substantial security costs (vaults, alarms, surveillance, armed protection), one might as well bear those costs for precious metals instead, thereby eliminating the technological complexity inherent to crypto custody.

Notably, any such capital flight would likely remain invisible in public data. Holders motivated by security concerns would, almost by definition, execute these transitions discreetly, precisely because, to them, security through obscurity appears paramount given the gravity and persistence of the wrench attack threat. This suggests the observable trend may significantly understate the actual magnitude of the shift. Additionally, that explanation also suggests an \href{https://en.wikipedia.org/wiki/Anosognosia}{\textit{anosognostic}} mechanism: the problem, by its nature, causes individuals to prefer, for their own security, not to talk about it, which impedes accurate diagnose and potential treatments.

This dynamic represents a significant tailwind for Great Wall. Rather than abandoning cryptocurrency, holders can retain the benefits of digital assets while neutralizing the security concerns that would otherwise drive them toward traditional stores of value.

\subsection{Competitive Landscape}
\begin{itemize}
  \item \textbf{Direct Competition:} Limited due to anonymous marketplace complexity
  \item \textbf{Indirect Competition:} Traditional cloud computing lacks privacy features
  \item \textbf{Barriers to Entry:} Trust and reputation system creates moat
  \item \textbf{First-Mover Advantage:} Early provider network difficult to replicate
\end{itemize}
