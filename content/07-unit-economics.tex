% 07-unit-economics.tex - Unit economics summary

\section{Unit Economics Summary}

\begin{table}[H]
\centering
\begin{tabularx}{\linewidth}{l Y}
\toprule
Metric & Subscriptions \\\midrule
Average Revenue (Marketplace) & \$\numint{\markWeightedAvgAnnual}/year \\
Gross Margin & \num{\subGrossMargin}\% \\
CAC & \$\num{\cacDigital} \\
LTV & \$\numint{\subLTV} \\
LTV:CAC Ratio & \numint{\ltvCacRatio}:1 \\
Payback Period & \num{\subPaybackMonths} months \\
\bottomrule
\end{tabularx}
\end{table}

\subsection{Gross Margin Breakdown}
Our \num{\subGrossMargin}\% gross margin is derived bottom-up from explicit COGS components, positioning us conservatively within the 60--80\% marketplace benchmark range\cite{andreessen2020}:

\begin{table}[H]
\centering
\begin{tabularx}{\linewidth}{l Y X}
\toprule
COGS Component & \% of Net Rev & Description \\\midrule
Payment Processing & \num{\cogsPaymentProcessingPercent}\% & \num{\paymentProcessingFeeGross}\% on GMV $\div$ \num{\markPlatformFeePercent}\% take rate \\
Infrastructure & \num{\cogsInfraPercent}\% & Matching platform, hosting, databases \\
Support/Disputes & \num{\cogsSupportPercent}\% & Customer support, dispute arbitration \\
Trust \& Safety & \num{\cogsTrustSafetyPercent}\% & Fraud prevention, security, compliance \\\midrule
\textbf{Total COGS} & \textbf{\num{\totalCogsPercent}\%} & \\
\textbf{Gross Margin} & \textbf{\num{\subGrossMargin}\%} & = 100\% $-$ COGS \\
\bottomrule
\end{tabularx}
\end{table}

\subsection{Key Economic Insights}
\begin{itemize}
    \item \textbf{Subscription Economics:} Our \num{\subGrossMargin}\% gross margin reflects the capital-light marketplace model. COGS are dominated by payment processing (\num{\cogsPaymentProcessingPercent}\% of net revenue), with additional costs for infrastructure, support, and trust/safety operations. We plan to incentivize Lightning Network adoption to reduce payment processing costs over time.
    
    \item \textbf{Competitive Platform Fee:} \markPlatformFeePercent\% take rate aligns with industry leaders (Uber 25\%, Fiverr 20\%, Amazon 15-45\%)\cite{uber2023,fiverr2023,amazon2024}
    
    \item \textbf{Customer Acquisition Efficiency:} Our platform design creates natural viral growth through provider incentives. Industry benchmarks show marketplaces typically spend \$35-50 per customer acquired\cite{andreessen2020}, while our blended CAC is just \$\numint{\cacDigital}. This efficiency stems from providers actively recruiting clients to increase their own revenue---a dynamic documented in successful platforms like Uber (drivers recruiting riders) and Airbnb (hosts encouraging bookings)\cite{kumar2007,nfx2018}. Includes the free gift effect (\num{\freeMerchanCacReductionPercent}\% CAC reduction) observed in case studies\cite{snappy2024}.
\end{itemize}

\subsubsection{Provider-Driven Growth Economics}
Academic research on two-sided platforms demonstrates that when supply-side participants directly benefit from demand growth, customer acquisition costs can decrease by 40-70\% compared to traditional advertising\cite{parker2016}. In our model, providers who recruit just one additional client increase their monthly profit by \$\num{\patrickProfitBasic}--\num{\patrickProfitGolden} (depending on tier), creating powerful organic growth incentives. This dynamic explains why our \$\numint{\marketingBudgetYearOne/1000}--\numint{\marketingBudgetYearTwo/1000}k annual marketing budgets achieve growth rates comparable to marketplaces spending \$200-400k\cite{bvp2023}.

\subsection{Cohort Economics}
\begin{itemize}
    \item \textbf{Customer Lifetime:} Average \numfpeval{round(\ltvYearsUsed, 1)} years (capped at \num{\ltvCapYears} years)
    \item \textbf{Churn Improvement:} From \num{\churnYearOne}\% to \num{\churnYearThree}\% annually
    \item \textbf{Revenue Retention:} Strong unit retention with growing revenue per user through tier upgrades
\end{itemize}

\subsection{LTV Calculation Transparency}
\begin{table}[H]
\centering
\begin{tabularx}{\linewidth}{l Y l}
\toprule
Component & Value & Calculation \\\midrule
Annual Platform Revenue/Sub & \$\num{\markWeightedAvgAnnual} & Weighted avg across tiers \\
Gross Margin & \num{\subGrossMargin}\% & Derived from COGS \\
Annual Gross Profit/Sub & \$\num{\subAnnualGrossProfit} & Revenue $\times$ Margin \\
Customer Lifetime & \num{\ltvYearsUsed} years & min(1/churn, \num{\ltvCapYears} cap) \\
\textbf{LTV} & \textbf{\$\numint{\subLTV}} & GP/year $\times$ Lifetime \\
\bottomrule
\end{tabularx}
\end{table}

\subsection{Sensitivity Analysis}

Unit economics remain robust across a range of input assumptions. The following analysis tests $\pm$\num{\sensitivityDelta}\% variations in key drivers.

\begin{table}[H]
\centering
\caption{LTV:CAC Sensitivity to Individual Input Changes}
\begin{tabularx}{\linewidth}{l Y Y Y Y}
\toprule
\textbf{Scenario} & \textbf{Input Change} & \textbf{LTV:CAC} & \textbf{Payback} & \textbf{vs. Base} \\\midrule
\textbf{Base Case} & --- & \numint{\ltvCacRatio}:1 & \num{\subPaybackMonths}~mo & --- \\\midrule
\multicolumn{5}{l}{\textit{CAC Sensitivity}} \\
CAC +\num{\sensitivityDelta}\% & \$\num{\sensCAChigh} & \numint{\ltvCacHighCAC}:1 & \num{\paybackHighCAC}~mo & \num{\pctChangeHighCAC}\% \\
CAC $-$\num{\sensitivityDelta}\% & \$\num{\sensCAClow} & \numint{\ltvCacLowCAC}:1 & \num{\paybackLowCAC}~mo & +\num{\pctChangeLowCAC}\% \\\midrule
\multicolumn{5}{l}{\textit{Churn Sensitivity (LTV capped at \num{\ltvCapYears} years)}} \\
Churn +\num{\sensitivityDelta}\% & \num{\sensChurnHighPercent}\%/yr & \numint{\ltvCacHighChurn}:1 & \num{\subPaybackMonths}~mo & 0\%* \\
Churn $-$\num{\sensitivityDelta}\% & \num{\sensChurnLowPercent}\%/yr & \numint{\ltvCacLowChurn}:1 & \num{\subPaybackMonths}~mo & 0\%* \\\midrule
\multicolumn{5}{l}{\textit{Gross Margin Sensitivity}} \\
Margin +\num{\sensitivityDelta}\% & \num{\sensGrossMarginHigh}\% & \numint{\ltvCacHighMargin}:1 & \num{\paybackHighMargin}~mo & +\num{\pctChangeHighMargin}\% \\
Margin $-$\num{\sensitivityDelta}\% & \num{\sensGrossMarginLow}\% & \numint{\ltvCacLowMargin}:1 & \num{\paybackLowMargin}~mo & \num{\pctChangeLowMargin}\% \\
\bottomrule
\end{tabularx}
\end{table}
\textit{*Churn changes have no LTV impact because theoretical lifetime (\num{\theoreticalLifetimeYears} years) exceeds the \num{\ltvCapYears}-year cap in all scenarios.}

\subsubsection{Combined Scenario Analysis}

\begin{table}[H]
\centering
\caption{Stress Test and Upside Scenarios}
\begin{tabularx}{\linewidth}{l X Y Y Y}
\toprule
\textbf{Scenario} & \textbf{Assumptions} & \textbf{LTV:CAC} & \textbf{Payback} & \textbf{vs. Base} \\\midrule
\textbf{Base Case} & Current projections & \numint{\ltvCacRatio}:1 & \num{\subPaybackMonths}~mo & --- \\
\textbf{Stress Test} & CAC +\num{\sensitivityDelta}\%, Churn +\num{\sensitivityDelta}\%, Margin $-$\num{\sensitivityDelta}\% & \numint{\stressLtvCac}:1 & \num{\stressPayback}~mo & \num{\pctChangeStress}\% \\
\textbf{Upside} & CAC $-$\num{\sensitivityDelta}\%, Churn $-$\num{\sensitivityDelta}\%, Margin +\num{\sensitivityDelta}\% & \numint{\upsideLtvCac}:1 & \num{\upsidePayback}~mo & +\num{\pctChangeUpside}\% \\
\bottomrule
\end{tabularx}
\end{table}

\subsubsection{Key Takeaways}

\begin{itemize}
    \item \textbf{Downside Protection:} Even under stress test conditions (all inputs move adversely by \num{\sensitivityDelta}\%), LTV:CAC remains \numint{\stressLtvCac}:1---well above the 3:1 threshold for healthy unit economics.
    
    \item \textbf{CAC is the Primary Driver:} A \num{\sensitivityDelta}\% CAC increase reduces LTV:CAC by \num{\pctChangeHighCAC}\%, making customer acquisition efficiency our most sensitive lever.
    
    \item \textbf{Churn Impact is Capped:} Due to our conservative \num{\ltvCapYears}-year LTV cap, churn variations within $\pm$\num{\sensitivityDelta}\% have no impact on calculated LTV. This cap provides natural downside protection.
    
    \item \textbf{Margin Resilience:} Even with \num{\sensitivityDelta}\% gross margin compression, LTV:CAC of \numint{\ltvCacLowMargin}:1 supports aggressive growth investment.
\end{itemize}
