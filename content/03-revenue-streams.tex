% 03-revenue-streams.tex - Detailed revenue model with CAC optimization

\section{Enhanced Business Model with CAC Optimization}

\subsection{Subscription Pricing (Based on Computation Economics)}
\begin{table}[H]
\centering
\caption{Anonymous Computation Marketplace Economics}
\resizebox{\textwidth}{!}{%
\begin{tabularx}{1.2\linewidth}{l Y Y Y Y Y Y Y Y}
\toprule
Tier & Delay (Hours) & Price & Mix & Provider Cost & Provider GP & Marketplace Rev \\\midrule
Basic  & \numint{\computeHoursBasic} & \num{\subBasicPrice}  & \numint{\subBasicMix}\% & \numfpeval{\patrickCostBasic} & \numfpeval{\patrickProfitBasic} & \numfpeval{\markRevenueBasic} \\
Medium  & \numint{\computeHoursMedium} & \num{\subMediumPrice} & \numint{\subMediumMix}\% & \numfpeval{\patrickCostMedium} & \numfpeval{\patrickProfitMedium} & \numfpeval{\markRevenueMedium} \\
Professional & \numint{\computeHoursProfessional} & \num{\subProfessionalPrice} & \numint{\subProfessionalMix}\% & \numfpeval{\patrickCostProfessional} & \numfpeval{\patrickProfitProfessional} & \numfpeval{\markRevenueProfessional} \\
Golden  & \numint{\computeHoursGolden} & \num{\subGoldenPrice} & \numint{\subGoldenMix}\% & \numfpeval{\patrickCostGolden} & \numfpeval{\patrickProfitGolden} & \numfpeval{\markRevenueGolden} \\\midrule
\textbf{Weighted Avg} &  & \textbf{\numfpeval{\subWeightedAvgPrice}} & \textbf{\num{100}}\% &  &  & \textbf{\numfpeval{\markWeightedAvgMonthly}} \\
\bottomrule
\end{tabularx}
}
\end{table}
\textit{Note: Provider's costs based on \numint{\computePowerWatts}W @ \$\num{\electricityCostPerKwh}/kWh, \numfpeval{\weeklyRunsPerMonth} runs/month. All values in USD/month.}

\subsection{Marketplace Economics and Competitive Analysis}
Our platform operates with a \markPlatformFeePercent\% commission rate, positioning us competitively within the marketplace landscape:

\begin{itemize}
  \item \textbf{Patrick's Progressive Markup (\numfpeval{\patrickMarkupBasic}--\numfpeval{\patrickMarkupGolden}x):} 
  \begin{itemize}
    \item Basic (\numfpeval{\patrickMarkupBasic}x): Entry-level commitments with quick turnarounds; lower markup to seed supply and onboard providers
    \item Medium (\numfpeval{\patrickMarkupMedium}x): Day-scale jobs add coordination and opportunity costs; moderate markup reflects added diligence
    \item Professional (\numfpeval{\patrickMarkupProfessional}x): Multi-day (48h) runs lock capacity and raise reliability risk; premium markup prices scarcity
    \item Golden (\numfpeval{\patrickMarkupGolden}x): Week-long workloads require sustained resource dedication and scheduling discipline; highest balanced markup secures dependable supply
  \end{itemize}
  \item \textbf{Platform Fee Benchmarks:}
  \begin{itemize}
    \item Our platform: \markPlatformFeePercent\% - includes full-service anonymous matchmaking, reputation system, and dispute resolution
    \item Uber: ~25\% commission\cite{uber2023}
    \item Airbnb: ~15\% total fees\cite{airbnb2023}  
    \item Amazon Marketplace: 15-45\% depending on category\cite{amazon2024}
    \item Fiverr: 20\% from sellers\cite{fiverr2023}
    \item Upwork: 20\% for first \$500, then 10\%\cite{upwork2023}
  \end{itemize}
  
  \item \textbf{Why Our Economics Work:} Unlike traditional marketplaces that spend 15-30\% of revenue on customer acquisition\cite{andreessen2020}, our model creates natural viral growth. Providers actively recruit clients to increase their own revenue, functioning as an unpaid but highly motivated sales force. This alignment means we achieve similar growth with marketing budgets of just \$80-150k annually rather than the \$200-400k typical for our revenue scale.
  
  \item \textbf{Provider Economics Remain Attractive:} Even at 28\% platform fee, providers earn \patrickMarkupBasic-\patrickMarkupGolden  their electricity costs depending on tier, creating sustainable incentives for participation. Academic research shows that successful two-sided platforms maintain take rates between 20-30\% when providing high-value services\cite{rochet2003,hagiu2015}.
\end{itemize}


\subsection{Merchandise Economics (CAC Offset Strategy)}
\begin{table}[H]
\centering
\begin{tabularx}{\linewidth}{l Y Y Y}
\toprule
Product Category & Price (USD) & Margin \cite{printful2023} & Profit/Unit \\\midrule
T-shirts & \num{\merchTshirtPrice} & \numint{\merchTshirtMargin}\% & \numfpeval{\merchTshirtPrice * \merchTshirtMargin / 100} \\
Caps/Hats & \num{\merchCapPrice} & \numint{\merchCapMargin}\% & \numfpeval{\merchCapPrice * \merchCapMargin / 100} \\
Mugs & \num{\merchMugPrice} & \numint{\merchMugMargin}\% & \numfpeval{\merchMugPrice * \merchMugMargin / 100} \\
Stickers/Decals & \num{\merchStickerPrice} & \numint{\merchStickerMargin}\% & \numfpeval{\merchStickerPrice * \merchStickerMargin / 100} \\
Hoodies/Coats & \num{\merchHoodiePrice} & \numint{\merchHoodieMargin}\% & \numfpeval{\merchHoodiePrice * \merchHoodieMargin / 100} \\
Backpacks & \num{\merchBackpackPrice} & \numint{\merchBackpackMargin}\% & \numfpeval{\merchBackpackPrice * \merchBackpackMargin / 100} \\\midrule
\textbf{Average basket} & \textbf{\num{\merchAvgPrice}} & \textbf{\numint{\merchAvgMargin}\%} & \textbf{\numfpeval{\merchAvgProfit}} \\
\bottomrule
\end{tabularx}
\end{table}
\textit{\num{\merchAttachRate}\% of subscription customers purchase merchandise, reducing effective CAC by \$\numfpeval{\merchOffsetSubsCalc}}\cite{shopify2024}

\noindent The high attachment rate reflects deterrent nature of the security-branded merchandise. By communicating preparedness and coercion resistance ("not an easy target"), it provides utility beyond brand affinity, increasing purchase propensity among security-conscious users and reinforcing a community norm around visible commitment to security.

\textit{Note: Merchandise purchased by Free (self-hosted) users is not included in revenue/CAC projections for simplicity.}
