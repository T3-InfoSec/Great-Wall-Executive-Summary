% 06-valuation-analysis.tex - Valuation methodology and projections

\section{Valuation Analysis}

\subsection{Multiple-Based Valuation}
\begin{table}[H]
\centering
\begin{tabularx}{\linewidth}{l Y Y Y Y}
\toprule
Component & Multiple\cite{highalpha2024,openview2023} & Y1 Value & Y2 Value & Y3 Value \\\midrule
Subscription Exit ARR & \num{\arrMultiple}x & \numint{\subARRYearOne * \arrMultiple} & \numint{\subARRYearTwo * \arrMultiple} & \numint{\subARRYearThree * \arrMultiple} \\\midrule
\textbf{Total Valuation} &  & \textbf{\numint{\subARRYearOne * \arrMultiple}} & \textbf{\numint{\subARRYearTwo * \arrMultiple}} & \textbf{\numint{\subARRYearThree * \arrMultiple}} \\
\bottomrule
\end{tabularx}
\end{table}

\subsection{Investment Timeline and Valuation Progression}
\begin{table}[H]
\centering
\begin{tabularx}{\linewidth}{l l r r L}
\toprule
Stage & Timing & Funding & Valuation & Basis \\\midrule
\textbf{Seed} & Month 0 & \$\numint{round(\seedAmount/1000,0)}K & \$\numfpeval{round(\seedValuation/1000000,1)}M & Market comparables* \\
\textbf{Series A} & Month 18 & \$\numint{\seriesAAmount/1000000}M & \$\numfpeval{\seriesAValuation/1000000}M & Financing-driven; supported by forward net ARR trajectory, CAC efficiency, and movement-led GTM \\
\textbf{Target} & Year 3 & 0 & \$\num{\targetValLow}--\num{\targetValHigh}M & \$\numfpeval{round(\subARRYearThree/1000000,1)}M $\times$ \numfpeval{\targetARRMultiple}--\num{\arrMultiple} (ARR mult.) \\
\textbf{Optimistic} & Year 3 & 0 & \$\num{\optimisticValLow}--\num{\optimisticValHigh}M & Premium multiples \\
\bottomrule
\end{tabularx}
\end{table}
\textit{*Pre-revenue valuation based on team, TAM, and marketplace model - not formulaic}

\par\noindent\textit{ARR figures refer to net platform revenue (take rate), not GMV.}

\paragraph{Series A valuation rationale}
The Series A valuation is supported by a combination of capital efficiency, de-risked execution, and a defensible go-to-market engine:
\begin{itemize}
    \item \textbf{Protocol innovation + movement as GTM:} Brand-led acquisition (organic inbound, press, community), conversion uplift (deterrence narrative), and retention uplift (identity/community). We track this via CAC, conversion, churn, and referral rate.
    \item \textbf{Capital efficiency:} CAC of \$\numint{\cacDigital} with \num{\subPaybackMonths}-month payback supports scalable growth with modest marketing spend.
    \item \textbf{De-risking milestones:} MVP/beta traction and a clear path to breakeven by Month \numint{\breakevenMonth} reduce downside risk ahead of scale.
\end{itemize}

\subsection{Growth \& Unit Economics Supporting Valuation}
\begin{itemize}
    \item \textbf{Accelerating ARR Growth:} Proven scaling from \numfpeval{round(\growthRateYearOneTwo, 1)}\% (Y1-Y2) to \numfpeval{round(\growthRateYearTwoThree, 1)}\% (Y2-Y3).
    
    \item \textbf{Exceptional Unit Economics:} A powerful \numint{\subLTV/(\cacDigital)}:1 LTV:CAC ratio enables our aggressive growth strategy.
    
    \item \textbf{Aggressive, Data-Driven Acquisition:} Scaling marketing from \$\numint{\marketingBudgetYearOne} (Y1) to \$\numint{\marketingBudgetYearThree} (Y3). This investment is fueled by our proven \numint{\subLTV/(\cacDigital)}:1 LTV:CAC.
    
    \item \textbf{Provider Growth Flywheel:} Organic alignment, amplified by targeted referral and rev-share programs, to rapidly scale network density.
    
    \item \textbf{Strategic Market Position:} Competitive \markPlatformFeePercent\% take rate with a clear strategy to capture \num{\targetShareSubs}\% market share.
\end{itemize}
