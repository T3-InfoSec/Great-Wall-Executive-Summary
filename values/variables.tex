% values/variables.tex - Fully Softcoded Financial Model
% All derived values computed from base inputs for internal consistency
% ============================================================================

% ============================================================================
% SECTION 1: TIMELINE DATES (Hardcoded milestones)
% ============================================================================
\newcommand{\mvpDate}{Feb 28, 2026}
\newcommand{\betaTenDate}{Mar 10, 2026}
\newcommand{\betaEndDate}{May 2026}
\newcommand{\scaleMktDate}{Jun 2026}
\newcommand{\kCustomersDate}{Sep 2026}

% Timeline months for detailed schedule
\newcommand{\marketplaceBuildStartMonth}{1}
\newcommand{\marketplaceBuildEndMonth}{3}
\newcommand{\betaLaunchStartMonth}{4}
\newcommand{\betaLaunchEndMonth}{6}
\newcommand{\marketplaceScaleStartMonth}{7}
\newcommand{\marketplaceScaleEndMonth}{12}

% Beta milestones
\newcommand{\betaFirstClients}{10}
\newcommand{\betaEndClients}{50}
\newcommand{\betaUsers}{1000}

% Provider referral program parameters
\newcommand{\providerCommissionPercent}{20}
\newcommand{\referralProgramEndCustomers}{20000}

% ============================================================================
% SECTION 2: CORE BUSINESS PARAMETERS (Base inputs)
% ============================================================================

% --- Computation Economics ---
\newcommand{\computePowerWatts}{350}
\newcommand{\electricityCostPerKwh}{0.12}
\newcommand{\weeklyRunsPerMonth}{\fpeval{30/7}} % average weeks in a month

% --- Tier Configuration (hours per job) ---
\newcommand{\computeHoursBasic}{2}
\newcommand{\computeHoursMedium}{24}
\newcommand{\computeHoursProfessional}{48}
\newcommand{\computeHoursGolden}{168}

% --- Tier Mix (must sum to 100) ---
\newcommand{\subBasicMix}{35}
\newcommand{\subMediumMix}{40}
\newcommand{\subProfessionalMix}{15}
\newcommand{\subGoldenMix}{10}

% --- Provider Markup Multipliers ---
\newcommand{\patrickMarkupBasic}{2.5}
\newcommand{\patrickMarkupMedium}{3}
\newcommand{\patrickMarkupProfessional}{3.5}
\newcommand{\patrickMarkupGolden}{5}

% --- Platform Fee ---
\newcommand{\markPlatformFeePercent}{28}

% --- Churn Rates (annual %) ---
\newcommand{\churnYearOne}{5}
\newcommand{\churnYearTwo}{4.5}
\newcommand{\churnYearThree}{4}

% --- Payment Processing (on gross transaction volume) ---
\newcommand{\paymentProcessingFeeGross}{2.9}

% --- LTV Parameters ---
\newcommand{\ltvCapYears}{7}

% --- Revenue Timing (avg months new subs contribute in acquisition year) ---
\newcommand{\avgRevenueMonthsYearOne}{6.5}

% ============================================================================
% SECTION 3: PROVIDER ECONOMICS (Derived from computation parameters)
% ============================================================================

% Provider electricity cost per run = (watts/1000) * hours * $/kWh * runs/month
\newcommand{\patrickCostBasic}{\fpeval{\computePowerWatts/1000 * \computeHoursBasic * \electricityCostPerKwh * \weeklyRunsPerMonth}}
\newcommand{\patrickCostMedium}{\fpeval{\computePowerWatts/1000 * \computeHoursMedium * \electricityCostPerKwh * \weeklyRunsPerMonth}}
\newcommand{\patrickCostProfessional}{\fpeval{\computePowerWatts/1000 * \computeHoursProfessional * \electricityCostPerKwh * \weeklyRunsPerMonth}}
\newcommand{\patrickCostGolden}{\fpeval{\computePowerWatts/1000 * \computeHoursGolden * \electricityCostPerKwh * \weeklyRunsPerMonth}}
% HC: Expected values
\newcommand{\HCpatrickCostBasic}{0.36}
\newcommand{\HCpatrickCostMedium}{4.32}
\newcommand{\HCpatrickCostProfessional}{8.64}
\newcommand{\HCpatrickCostGolden}{30.24}

% Provider gross revenue = cost * markup
\newcommand{\patrickGrossBasic}{\fpeval{\patrickCostBasic * \patrickMarkupBasic}}
\newcommand{\patrickGrossMedium}{\fpeval{\patrickCostMedium * \patrickMarkupMedium}}
\newcommand{\patrickGrossProfessional}{\fpeval{\patrickCostProfessional * \patrickMarkupProfessional}}
\newcommand{\patrickGrossGolden}{\fpeval{\patrickCostGolden * \patrickMarkupGolden}}
% HC: Expected values
\newcommand{\HCpatrickGrossBasic}{0.90}
\newcommand{\HCpatrickGrossMedium}{12.96}
\newcommand{\HCpatrickGrossProfessional}{30.24}
\newcommand{\HCpatrickGrossGolden}{151.20}

% Provider profit = gross - cost
\newcommand{\patrickProfitBasic}{\fpeval{\patrickGrossBasic - \patrickCostBasic}}
\newcommand{\patrickProfitMedium}{\fpeval{\patrickGrossMedium - \patrickCostMedium}}
\newcommand{\patrickProfitProfessional}{\fpeval{\patrickGrossProfessional - \patrickCostProfessional}}
\newcommand{\patrickProfitGolden}{\fpeval{\patrickGrossGolden - \patrickCostGolden}}
% HC: Expected values
\newcommand{\HCpatrickProfitBasic}{0.54}
\newcommand{\HCpatrickProfitMedium}{8.64}
\newcommand{\HCpatrickProfitProfessional}{21.60}
\newcommand{\HCpatrickProfitGolden}{120.96}

% Provider monthly profit per client (min/max across tiers)
% Used in unit economics to show provider incentive range
\newcommand{\providerMonthlyProfitMin}{\fpeval{round(\patrickProfitBasic, 0)}}
\newcommand{\providerMonthlyProfitMax}{\fpeval{round(\patrickProfitGolden, 0)}}

% ============================================================================
% SECTION 4: SUBSCRIPTION PRICING (Derived from provider economics + platform fee)
% ============================================================================

% Full subscription price = provider gross / (1 - platform fee)
% This ensures: price * platformFee = markRevenue, and price - markRevenue = patrickGross
\newcommand{\subBasicPrice}{\fpeval{\patrickGrossBasic / (1 - \markPlatformFeePercent/100)}}
\newcommand{\subMediumPrice}{\fpeval{\patrickGrossMedium / (1 - \markPlatformFeePercent/100)}}
\newcommand{\subProfessionalPrice}{\fpeval{\patrickGrossProfessional / (1 - \markPlatformFeePercent/100)}}
\newcommand{\subGoldenPrice}{\fpeval{\patrickGrossGolden / (1 - \markPlatformFeePercent/100)}}
% HC: Expected values
\newcommand{\HCsubBasicPrice}{1.25}
\newcommand{\HCsubMediumPrice}{18.00}
\newcommand{\HCsubProfessionalPrice}{42.00}
\newcommand{\HCsubGoldenPrice}{210.00}

% Mark's (platform) revenue per tier = price * platform fee
\newcommand{\markRevenueBasic}{\fpeval{\subBasicPrice * \markPlatformFeePercent/100}}
\newcommand{\markRevenueMedium}{\fpeval{\subMediumPrice * \markPlatformFeePercent/100}}
\newcommand{\markRevenueProfessional}{\fpeval{\subProfessionalPrice * \markPlatformFeePercent/100}}
\newcommand{\markRevenueGolden}{\fpeval{\subGoldenPrice * \markPlatformFeePercent/100}}
% HC: Expected values
\newcommand{\HCmarkRevenueBasic}{0.35}
\newcommand{\HCmarkRevenueMedium}{5.04}
\newcommand{\HCmarkRevenueProfessional}{11.76}
\newcommand{\HCmarkRevenueGolden}{58.80}

% Weighted average prices (full price to customer)
\newcommand{\subWeightedAvgPrice}{\fpeval{
    \subBasicPrice * \subBasicMix/100 +
    \subMediumPrice * \subMediumMix/100 +
    \subProfessionalPrice * \subProfessionalMix/100 +
    \subGoldenPrice * \subGoldenMix/100
}}
\newcommand{\HCsubWeightedAvgPrice}{34.94}

% Weighted average platform revenue (Mark's monthly revenue per subscriber)
\newcommand{\markWeightedAvgMonthly}{\fpeval{
    \markRevenueBasic * \subBasicMix/100 +
    \markRevenueMedium * \subMediumMix/100 +
    \markRevenueProfessional * \subProfessionalMix/100 +
    \markRevenueGolden * \subGoldenMix/100
}}
\newcommand{\HCmarkWeightedAvgMonthly}{9.78}

% Annual platform revenue per subscriber
\newcommand{\markWeightedAvgAnnual}{\fpeval{\markWeightedAvgMonthly * 12}}
\newcommand{\HCmarkWeightedAvgAnnual}{117.39}

% Monthly revenue per sub (alias for clarity)
\newcommand{\monthlyRevPerSub}{\markWeightedAvgMonthly}

% ============================================================================
% SECTION 5: COST OF GOODS SOLD (COGS) - Bottom-Up Estimation
% ============================================================================
% COGS expressed as percentage of NET REVENUE (platform's take rate portion)
% Sources: Accel/Mattermark marketplace benchmarks (60-80% gross margin range)
%          OpenMetal SaaS infrastructure benchmarks (8-15% infra COGS)

% --- Payment Processing (derived) ---
% Payment processing charged on gross volume, expressed as % of net revenue
% Formula: grossFee% / takeRate% = effective % of net revenue
\newcommand{\cogsPaymentProcessingPercent}{\fpeval{\paymentProcessingFeeGross / (\markPlatformFeePercent/100)}}
% = 2.9 / 0.28 = 10.36%

% --- Infrastructure (matching platform, reputation system, hosting) ---
% Benchmark: 8-15% for SaaS; using 10% as early-stage estimate
% Includes: AWS/cloud compute, database, CDN, monitoring
\newcommand{\cogsInfraPercent}{10}

% --- Customer Support & Dispute Resolution ---
% Marketplace-specific: arbitration, refund processing, user support
% Estimate based on support cost as % of revenue for service marketplaces
\newcommand{\cogsSupportPercent}{8}

% --- Trust & Safety ---
% Fraud prevention, identity verification, security audits, compliance
% Critical for anonymous marketplace; conservative estimate
\newcommand{\cogsTrustSafetyPercent}{5}

% --- Total COGS (derived) ---
\newcommand{\totalCogsPercent}{\fpeval{
    \cogsPaymentProcessingPercent + 
    \cogsInfraPercent + 
    \cogsSupportPercent + 
    \cogsTrustSafetyPercent
}}

% --- Gross Margin (derived) ---
\newcommand{\subGrossMargin}{\fpeval{100 - \totalCogsPercent}}
% Expected result: ~67% (within 60-80% marketplace benchmark range)

% ============================================================================
% SECTION 6: CUSTOMER ACQUISITION COST (CAC)
% ============================================================================

% Base CAC before gift effect
\newcommand{\cacDigitalWithoutFreeMerchan}{20}
\newcommand{\cacContentWithoutFreeMerchan}{18}

% Free gift parameters
\newcommand{\freeMerchanCacReductionPercent}{20}
\newcommand{\freeMerchanCost}{5}
\newcommand{\freeMerchAttachRate}{75}

% Effective CAC = (base CAC * (1 - reduction%)) + gift cost
\newcommand{\cacDigital}{\fpeval{\cacDigitalWithoutFreeMerchan * (1 - \freeMerchanCacReductionPercent/100) + \freeMerchanCost}}
\newcommand{\cacContent}{\fpeval{\cacContentWithoutFreeMerchan * (1 - \freeMerchanCacReductionPercent/100) + \freeMerchanCost}}
\newcommand{\HCcacDigital}{21.00}
\newcommand{\HCcacContent}{19.40}

% Budget allocation
\newcommand{\budgetDigital}{60000}
\newcommand{\budgetContent}{25000}

% ============================================================================
% SECTION 7: MARKETING BUDGETS & NEW SUBSCRIBER ACQUISITION
% ============================================================================

% Annual marketing budgets (base inputs)
\newcommand{\marketingBudgetYearOne}{80000}
\newcommand{\marketingBudgetYearTwo}{120000}
\newcommand{\marketingBudgetYearThree}{1000000}

% New subscribers acquired = marketing budget / CAC
\newcommand{\baseNewSubsYearOne}{\fpeval{round(\marketingBudgetYearOne / \cacDigital, 0)}}
\newcommand{\baseNewSubsYearTwo}{\fpeval{round(\marketingBudgetYearTwo / \cacDigital, 0)}}
\newcommand{\baseNewSubsYearThree}{\fpeval{round(\marketingBudgetYearThree / \cacDigital, 0)}}
\newcommand{\HCbaseNewSubsYearOne}{3810}
\newcommand{\HCbaseNewSubsYearTwo}{5714}
\newcommand{\HCbaseNewSubsYearThree}{47619}

% New subs (currently same as base; placeholder for future adjustments)
\newcommand{\newSubsYearOne}{\baseNewSubsYearOne}
\newcommand{\newSubsYearTwo}{\baseNewSubsYearTwo}
\newcommand{\newSubsYearThree}{\baseNewSubsYearThree}

% Average monthly acquisition rate (for reference)
\newcommand{\subsPerMonth}{\fpeval{\newSubsYearOne / 12}}

% ============================================================================
% SECTION 8: CUSTOMER GROWTH WITH CHURN (Derived - ORDER MATTERS)
% ============================================================================

% Year 1: No prior customers, total = new subs acquired
\newcommand{\totalSubsYearOne}{\newSubsYearOne}
\newcommand{\HCtotalSubsYearOne}{3810}

% Year 2: Retained from Y1 (after churn) + new Y2 subs
\newcommand{\retainedSubsYearTwo}{\fpeval{round(\totalSubsYearOne * (1 - \churnYearTwo/100), 0)}}
\newcommand{\totalSubsYearTwo}{\fpeval{\retainedSubsYearTwo + \newSubsYearTwo}}
\newcommand{\HCretainedSubsYearTwo}{3639}
\newcommand{\HCtotalSubsYearTwo}{9353}

% Year 3: Retained from Y2 (after churn) + new Y3 subs
\newcommand{\retainedSubsYearThree}{\fpeval{round(\totalSubsYearTwo * (1 - \churnYearThree/100), 0)}}
\newcommand{\totalSubsYearThree}{\fpeval{\retainedSubsYearThree + \newSubsYearThree}}
\newcommand{\HCretainedSubsYearThree}{8979}
\newcommand{\HCtotalSubsYearThree}{56598}

% ============================================================================
% SECTION 9: REVENUE CALCULATIONS (Derived from subscriber counts)
% ============================================================================

% Exit ARR = year-end subscribers * monthly platform revenue * 12
\newcommand{\subARRYearOne}{\fpeval{\totalSubsYearOne * \markWeightedAvgMonthly * 12}}
\newcommand{\subARRYearTwo}{\fpeval{\totalSubsYearTwo * \markWeightedAvgMonthly * 12}}
\newcommand{\subARRYearThree}{\fpeval{\totalSubsYearThree * \markWeightedAvgMonthly * 12}}
\newcommand{\HCsubARRYearOne}{447256}
\newcommand{\HCsubARRYearTwo}{1097949}
\newcommand{\HCsubARRYearThree}{6644039}

% Actual revenue collected (accounts for when subs join)
% Y1: New subs contribute avg months
\newcommand{\actualSubRevenueYearOne}{\fpeval{\newSubsYearOne * \markWeightedAvgMonthly * \avgRevenueMonthsYearOne}}
\newcommand{\HCactualSubRevenueYearOne}{242264}

% Y2: Retained subs (full 12 months) + new subs (avg months)
\newcommand{\actualSubRevenueYearTwo}{\fpeval{
    \retainedSubsYearTwo * \markWeightedAvgMonthly * 12 +
    \newSubsYearTwo * \markWeightedAvgMonthly * \avgRevenueMonthsYearOne
}}
\newcommand{\HCactualSubRevenueYearTwo}{790514}

% Y3: Retained subs (full 12 months) + new subs (avg months)
\newcommand{\actualSubRevenueYearThree}{\fpeval{
    \retainedSubsYearThree * \markWeightedAvgMonthly * 12 +
    \newSubsYearThree * \markWeightedAvgMonthly * \avgRevenueMonthsYearOne
}}
\newcommand{\HCactualSubRevenueYearThree}{4081958}

% Aliases for backward compatibility
\newcommand{\subRevenueYearOne}{\subARRYearOne}
\newcommand{\subRevenueYearTwo}{\subARRYearTwo}
\newcommand{\subRevenueYearThree}{\subARRYearThree}

% Gross transaction volume (GMV - for reference and COGS calculation)
\newcommand{\grossVolumeYearOne}{\fpeval{\actualSubRevenueYearOne / (\markPlatformFeePercent/100)}}
\newcommand{\grossVolumeYearTwo}{\fpeval{\actualSubRevenueYearTwo / (\markPlatformFeePercent/100)}}
\newcommand{\grossVolumeYearThree}{\fpeval{\actualSubRevenueYearThree / (\markPlatformFeePercent/100)}}
\newcommand{\HCgrossVolumeYearOne}{865227}
\newcommand{\HCgrossVolumeYearTwo}{2823264}
\newcommand{\HCgrossVolumeYearThree}{14578423}

% ============================================================================
% SECTION 10: COST OF GOODS SOLD - DOLLAR AMOUNTS (Derived)
% ============================================================================

% Total COGS in dollars = net revenue * COGS%
\newcommand{\totalCogsYearOne}{\fpeval{\actualSubRevenueYearOne * \totalCogsPercent/100}}
\newcommand{\totalCogsYearTwo}{\fpeval{\actualSubRevenueYearTwo * \totalCogsPercent/100}}
\newcommand{\totalCogsYearThree}{\fpeval{\actualSubRevenueYearThree * \totalCogsPercent/100}}
\newcommand{\HCtotalCogsYearOne}{80812}
\newcommand{\HCtotalCogsYearTwo}{263693}
\newcommand{\HCtotalCogsYearThree}{1361625}

% Gross Profit in dollars
\newcommand{\grossProfitYearOne}{\fpeval{\actualSubRevenueYearOne - \totalCogsYearOne}}
\newcommand{\grossProfitYearTwo}{\fpeval{\actualSubRevenueYearTwo - \totalCogsYearTwo}}
\newcommand{\grossProfitYearThree}{\fpeval{\actualSubRevenueYearThree - \totalCogsYearThree}}
\newcommand{\HCgrossProfitYearOne}{161451}
\newcommand{\HCgrossProfitYearTwo}{526821}
\newcommand{\HCgrossProfitYearThree}{2720334}

% COGS breakdown by component (for detailed reporting)
\newcommand{\cogsPaymentYearOne}{\fpeval{\actualSubRevenueYearOne * \cogsPaymentProcessingPercent/100}}
\newcommand{\cogsPaymentYearTwo}{\fpeval{\actualSubRevenueYearTwo * \cogsPaymentProcessingPercent/100}}
\newcommand{\cogsPaymentYearThree}{\fpeval{\actualSubRevenueYearThree * \cogsPaymentProcessingPercent/100}}

\newcommand{\cogsInfraYearOne}{\fpeval{\actualSubRevenueYearOne * \cogsInfraPercent/100}}
\newcommand{\cogsInfraYearTwo}{\fpeval{\actualSubRevenueYearTwo * \cogsInfraPercent/100}}
\newcommand{\cogsInfraYearThree}{\fpeval{\actualSubRevenueYearThree * \cogsInfraPercent/100}}

\newcommand{\cogsSupportYearOne}{\fpeval{\actualSubRevenueYearOne * \cogsSupportPercent/100}}
\newcommand{\cogsSupportYearTwo}{\fpeval{\actualSubRevenueYearTwo * \cogsSupportPercent/100}}
\newcommand{\cogsSupportYearThree}{\fpeval{\actualSubRevenueYearThree * \cogsSupportPercent/100}}

\newcommand{\cogsTrustYearOne}{\fpeval{\actualSubRevenueYearOne * \cogsTrustSafetyPercent/100}}
\newcommand{\cogsTrustYearTwo}{\fpeval{\actualSubRevenueYearTwo * \cogsTrustSafetyPercent/100}}
\newcommand{\cogsTrustYearThree}{\fpeval{\actualSubRevenueYearThree * \cogsTrustSafetyPercent/100}}

% ============================================================================
% SECTION 11: OPERATING EXPENSES (Excludes COGS - now separate)
% ============================================================================

% Team salaries (base inputs)
\newcommand{\teamSalariesYearOne}{240000}
\newcommand{\teamSalariesYearTwo}{600000}
\newcommand{\teamSalariesYearThree}{1200000}

% Infrastructure - ADDITIONAL beyond COGS (office, non-production tools)
% Note: Production infrastructure is now in COGS
\newcommand{\infrastructureYearOne}{10000}
\newcommand{\infrastructureYearTwo}{20000}
\newcommand{\infrastructureYearThree}{40000}

% Legal/Compliance (base inputs)
\newcommand{\legalComplianceYearOne}{20000}
\newcommand{\legalComplianceYearTwo}{30000}
\newcommand{\legalComplianceYearThree}{40000}

% Total OpEx (excluding COGS and marketing)
\newcommand{\totalOpexYearOne}{\fpeval{
    \teamSalariesYearOne + \infrastructureYearOne + \legalComplianceYearOne
}}
\newcommand{\totalOpexYearTwo}{\fpeval{
    \teamSalariesYearTwo + \infrastructureYearTwo + \legalComplianceYearTwo
}}
\newcommand{\totalOpexYearThree}{\fpeval{
    \teamSalariesYearThree + \infrastructureYearThree + \legalComplianceYearThree
}}
\newcommand{\HCtotalOpexYearOne}{270000}
\newcommand{\HCtotalOpexYearTwo}{650000}
\newcommand{\HCtotalOpexYearThree}{1280000}

% Total Operating Costs (OpEx + Marketing, excludes COGS)
\newcommand{\totalOperatingCostsYearOne}{\fpeval{\totalOpexYearOne + \marketingBudgetYearOne}}
\newcommand{\totalOperatingCostsYearTwo}{\fpeval{\totalOpexYearTwo + \marketingBudgetYearTwo}}
\newcommand{\totalOperatingCostsYearThree}{\fpeval{\totalOpexYearThree + \marketingBudgetYearThree}}
\newcommand{\HCtotalOperatingCostsYearOne}{350000}
\newcommand{\HCtotalOperatingCostsYearTwo}{770000}
\newcommand{\HCtotalOperatingCostsYearThree}{2280000}

% Monthly burn rate (COGS + OpEx + Marketing / 12)
\newcommand{\monthlyBurnYearOne}{\fpeval{(\totalCogsYearOne + \totalOpexYearOne + \marketingBudgetYearOne) / 12}}
\newcommand{\monthlyBurnYearTwo}{\fpeval{(\totalCogsYearTwo + \totalOpexYearTwo + \marketingBudgetYearTwo) / 12}}
\newcommand{\monthlyBurnYearThree}{\fpeval{(\totalCogsYearThree + \totalOpexYearThree + \marketingBudgetYearThree) / 12}}
\newcommand{\HCmonthlyBurnYearOne}{35901}
\newcommand{\HCmonthlyBurnYearTwo}{86141}
\newcommand{\HCmonthlyBurnYearThree}{303469}

% Net income (loss) = Revenue - COGS - OpEx - Marketing
\newcommand{\actualNetIncomeYearOne}{\fpeval{\actualSubRevenueYearOne - \totalCogsYearOne - \totalOpexYearOne - \marketingBudgetYearOne}}
\newcommand{\actualNetIncomeYearTwo}{\fpeval{\actualSubRevenueYearTwo - \totalCogsYearTwo - \totalOpexYearTwo - \marketingBudgetYearTwo}}
\newcommand{\actualNetIncomeYearThree}{\fpeval{\actualSubRevenueYearThree - \totalCogsYearThree - \totalOpexYearThree - \marketingBudgetYearThree}}

% Operating Income (EBIT) = Gross Profit - OpEx - Marketing
\newcommand{\operatingIncomeYearOne}{\fpeval{\grossProfitYearOne - \totalOpexYearOne - \marketingBudgetYearOne}}
\newcommand{\operatingIncomeYearTwo}{\fpeval{\grossProfitYearTwo - \totalOpexYearTwo - \marketingBudgetYearTwo}}
\newcommand{\operatingIncomeYearThree}{\fpeval{\grossProfitYearThree - \totalOpexYearThree - \marketingBudgetYearThree}}
\newcommand{\HCoperatingIncomeYearOne}{-188549}
\newcommand{\HCoperatingIncomeYearTwo}{-243179}
\newcommand{\HCoperatingIncomeYearThree}{440334}

% Operating Margin (for Rule of 40)
\newcommand{\operatingMarginYearThree}{\fpeval{\operatingIncomeYearThree / \actualSubRevenueYearThree * 100}}
\newcommand{\HCoperatingMarginYearThree}{10.79}

% ============================================================================
% SECTION 12: UNIT ECONOMICS & LTV
% ============================================================================
% NOTE: Churn values stored as percentages (e.g., 4.5 = 4.5%) not decimals
%       This avoids display rounding issues with values like 0.045

% Average annual churn as PERCENTAGE (for clean display)
% Formula: (5 + 4.5 + 4) / 3 = 4.5%
\newcommand{\avgAnnualChurnPercent}{\fpeval{(\churnYearOne + \churnYearTwo + \churnYearThree) / 3}}
\newcommand{\HCavgAnnualChurnPercent}{4.5}

% Theoretical customer lifetime = 100 / churn% (since churn is stored as percentage)
% Formula: 100 / 4.5 = 22.22 years
\newcommand{\theoreticalLifetimeYears}{\fpeval{100 / \avgAnnualChurnPercent}}
\newcommand{\HCtheoreticalLifetimeYears}{22.22}

% Actual lifetime used (capped)
\newcommand{\ltvYearsUsed}{\fpeval{min(\theoreticalLifetimeYears, \ltvCapYears)}}
\newcommand{\HCltvYearsUsed}{7.00}

% Annual gross profit per subscriber (uses derived gross margin)
\newcommand{\subAnnualGrossProfit}{\fpeval{\markWeightedAvgAnnual * \subGrossMargin/100}}
\newcommand{\HCsubAnnualGrossProfit}{78.23}

% Monthly gross profit per subscriber
\newcommand{\subMonthlyGrossProfit}{\fpeval{\markWeightedAvgMonthly * \subGrossMargin/100}}
\newcommand{\HCsubMonthlyGrossProfit}{6.52}

% Lifetime Value = annual gross profit * lifetime years
\newcommand{\subLTV}{\fpeval{\subAnnualGrossProfit * \ltvYearsUsed}}
\newcommand{\HCsubLTV}{547.62}

% Payback period (months) = CAC / monthly gross profit
\newcommand{\subPaybackMonths}{\fpeval{round(\cacDigital / \subMonthlyGrossProfit, 1)}}
\newcommand{\HCsubPaybackMonths}{3.2}

% LTV:CAC ratio
\newcommand{\ltvCacRatio}{\fpeval{round(\subLTV / \cacDigital, 0)}}
\newcommand{\HCltvCacRatio}{26}

% CAC reduction percent (placeholder for future viral/referral effects)
\newcommand{\cacReductionPercent}{0}

% ============================================================================
% SECTION 13: BREAKEVEN ANALYSIS
% ============================================================================

% Monthly fixed costs (OpEx + Marketing at Year 2 steady-state)
\newcommand{\monthlyFixedCostsYearTwo}{\fpeval{(\totalOpexYearTwo + \marketingBudgetYearTwo) / 12}}
\newcommand{\HCmonthlyFixedCostsYearTwo}{64167}

% Breakeven subscribers = monthly fixed costs / monthly gross profit per sub
\newcommand{\breakevenSubscribersRaw}{\fpeval{\monthlyFixedCostsYearTwo / \subMonthlyGrossProfit}}
\newcommand{\breakevenSubscribers}{\fpeval{round(\breakevenSubscribersRaw, -1)}}
\newcommand{\HCbreakevenSubscribers}{9840}

% Breakeven month estimate (when cumulative subs reach breakeven level)
% Approximate: subs grow from 0 to totalSubsYearOne in Y1, continue in Y2
% Linear interpolation: month = 12 + (breakeven - totalSubsYearOne) / monthly_growth_Y2
\newcommand{\monthlyGrowthYearTwo}{\fpeval{(\totalSubsYearTwo - \totalSubsYearOne) / 12}}
\newcommand{\breakevenMonthRaw}{\fpeval{12 + (\breakevenSubscribers - \totalSubsYearOne) / \monthlyGrowthYearTwo}}
\newcommand{\breakevenMonth}{\fpeval{round(\breakevenMonthRaw, 0)}}
\newcommand{\HCmonthlyGrowthYearTwo}{461.92}
\newcommand{\HCbreakevenMonth}{25}

% ============================================================================
% SECTION 14: FUNDING & RUNWAY
% ============================================================================

% Seed round
\newcommand{\seedAmount}{600000}
\newcommand{\seedEquity}{20}
\newcommand{\seedValuation}{3000000}
\newcommand{\seedPreMoneyNegotiated}{2400000}
\newcommand{\seedTeamSize}{3}

% Seed use allocation (for reference)
\newcommand{\seedUseSubPlatform}{50000}
\newcommand{\seedUseMarketing}{100000}
\newcommand{\seedUseOperations}{30000}
\newcommand{\seedUseWorkingCap}{20000}

% Series A
\newcommand{\seriesAAmount}{2000000}
\newcommand{\seriesAEquity}{25}
\newcommand{\seriesAValuation}{8000000}

% Cash positions
\newcommand{\cashAfterSeed}{\seedAmount}
\newcommand{\cashAfterSeriesA}{\seriesAAmount}

% Runway calculations (months)
\newcommand{\seedRunwayMonths}{\fpeval{round(\cashAfterSeed / \monthlyBurnYearOne, 0)}}
\newcommand{\seriesARunwayMonths}{\fpeval{round(\cashAfterSeriesA / \monthlyBurnYearTwo, 0)}}
\newcommand{\HCseedRunwayMonths}{17}
\newcommand{\HCseriesARunwayMonths}{23}

% ============================================================================
% SECTION 15: GROWTH RATES & VALUATION
% ============================================================================

% ARR growth rates (%)
\newcommand{\growthRateYearOneTwo}{\fpeval{(\subARRYearTwo - \subARRYearOne) / \subARRYearOne * 100}}
\newcommand{\growthRateYearTwoThree}{\fpeval{(\subARRYearThree - \subARRYearTwo) / \subARRYearTwo * 100}}
\newcommand{\HCgrowthRateYearOneTwo}{145.49}
\newcommand{\HCgrowthRateYearTwoThree}{505.13}

% Rule of 40 score (growth rate + operating margin)
% Now uses actual operating margin instead of gross margin
\newcommand{\ruleOfFortyScore}{\fpeval{\growthRateYearTwoThree + \operatingMarginYearThree}}
\newcommand{\HCruleOfFortyScore}{515.92}

% Valuation multiples
\newcommand{\arrMultiple}{3}
\newcommand{\targetARRMultiple}{2.5}
\newcommand{\optimisticARRMultiple}{4}
\newcommand{\yearTwoRevMultiple}{2.5}

% Target valuations (Year 3, in millions)
\newcommand{\targetValLow}{\fpeval{round(\subARRYearThree * \targetARRMultiple / 1000000, 0)}}
\newcommand{\targetValHigh}{\fpeval{round(\subARRYearThree * \arrMultiple / 1000000, 0)}}
\newcommand{\optimisticValLow}{\fpeval{round(\subARRYearThree * \arrMultiple / 1000000, 0)}}
\newcommand{\optimisticValHigh}{\fpeval{round(\subARRYearThree * \optimisticARRMultiple / 1000000, 0)}}
\newcommand{\HCtargetValLow}{17}
\newcommand{\HCtargetValHigh}{20}
\newcommand{\HCoptimisticValLow}{20}
\newcommand{\HCoptimisticValHigh}{27}

% ============================================================================
% SECTION 16: MARKET SIZING (TAM/SAM)
% ============================================================================

% TAM components (number of potential users globally)
\newcommand{\tamBitcoinUsers}{3000000}
\newcommand{\tamPasswordMgrUsers}{4000000}
\newcommand{\tamPhysicalVaultUsers}{500000}
\newcommand{\tamPrivateSecUsers}{1500000}

% Total TAM
\newcommand{\tamSubsGlobal}{\fpeval{\tamBitcoinUsers + \tamPasswordMgrUsers + \tamPhysicalVaultUsers + \tamPrivateSecUsers}}

% SAM (Serviceable Available Market) as % of TAM
\newcommand{\samPercentOfTam}{15}
\newcommand{\samSubs}{\fpeval{round(\tamSubsGlobal * \samPercentOfTam / 100, 0)}}

% Target market share
\newcommand{\targetShareSubs}{15}
\newcommand{\targetSubsCountk}{\fpeval{round(\samSubs * \targetShareSubs / 100 / 1000, 0)}}

% ============================================================================
% SECTION 17: MERCHANDISE (Excluded from projections - community engagement only)
% ============================================================================

\newcommand{\merchAttachRate}{0}
\newcommand{\merchAvgPrice}{28}
\newcommand{\merchAvgMargin}{48}
\newcommand{\merchAvgProfit}{\fpeval{\merchAvgPrice * \merchAvgMargin / 100}}
\newcommand{\merchOffsetSubsCalc}{0}

% Individual item pricing (for reference)
\newcommand{\merchTshirtPrice}{25}
\newcommand{\merchTshirtMargin}{50}
\newcommand{\merchHoodiePrice}{45}
\newcommand{\merchHoodieMargin}{40}
\newcommand{\merchCapPrice}{20}
\newcommand{\merchCapMargin}{45}
\newcommand{\merchMugPrice}{15}
\newcommand{\merchMugMargin}{55}
\newcommand{\merchStickerPrice}{5}
\newcommand{\merchStickerMargin}{70}
\newcommand{\merchBackpackPrice}{35}
\newcommand{\merchBackpackMargin}{45}

% ============================================================================
% SECTION 18: REFERENCE METRICS (For display/validation)
% ============================================================================

% Customers per thousand (for certain calculations)
\newcommand{\custPerThousandYearOne}{50}
\newcommand{\custPerThousandYearTwo}{50}
\newcommand{\custPerThousandYearThree}{50}

% Legacy compatibility - paymentProcessingFee alias
\newcommand{\paymentProcessingFee}{\paymentProcessingFeeGross}

% ============================================================================
% SECTION 19: VALIDATION HARDCODES FOR NEW VARIABLES
% ============================================================================

% Beta milestones (no HC needed - base inputs)
% Provider referral program (no HC needed - base inputs)

% Provider monthly profit range (derived)
\newcommand{\HCproviderMonthlyProfitMin}{1}
\newcommand{\HCproviderMonthlyProfitMax}{121}

% ============================================================================
% SECTION 20: SENSITIVITY ANALYSIS
% ============================================================================
% Models impact of ±20% changes in key input assumptions on unit economics
% All derived values follow the same calculation patterns as base case

% --- Sensitivity Parameters ---
\newcommand{\sensitivityDelta}{20}  % Percentage change to test (±20%)

% ============================================================================
% SCENARIO A: CAC SENSITIVITY
% ============================================================================

% CAC +20% scenario
\newcommand{\sensCAChigh}{\fpeval{\cacDigital * (1 + \sensitivityDelta/100)}}
\newcommand{\HCsensCAChigh}{25.20}

% CAC -20% scenario
\newcommand{\sensCAClow}{\fpeval{\cacDigital * (1 - \sensitivityDelta/100)}}
\newcommand{\HCsensCAClow}{16.80}

% LTV:CAC with high CAC (LTV unchanged, CAC increases)
\newcommand{\ltvCacHighCAC}{\fpeval{round(\subLTV / \sensCAChigh, 0)}}
\newcommand{\HCltvCacHighCAC}{22}

% LTV:CAC with low CAC (LTV unchanged, CAC decreases)
\newcommand{\ltvCacLowCAC}{\fpeval{round(\subLTV / \sensCAClow, 0)}}
\newcommand{\HCltvCacLowCAC}{33}

% Payback with high CAC
\newcommand{\paybackHighCAC}{\fpeval{round(\sensCAChigh / \subMonthlyGrossProfit, 1)}}
\newcommand{\HCpaybackHighCAC}{3.9}

% Payback with low CAC
\newcommand{\paybackLowCAC}{\fpeval{round(\sensCAClow / \subMonthlyGrossProfit, 1)}}
\newcommand{\HCpaybackLowCAC}{2.6}

% ============================================================================
% SCENARIO B: CHURN SENSITIVITY
% ============================================================================

% Churn +20% scenario (as percentage, e.g., 4.5% -> 5.4%)
\newcommand{\sensChurnHighPercent}{\fpeval{\avgAnnualChurnPercent * (1 + \sensitivityDelta/100)}}
\newcommand{\HCsensChurnHighPercent}{5.4}

% Churn -20% scenario (as percentage, e.g., 4.5% -> 3.6%)
\newcommand{\sensChurnLowPercent}{\fpeval{\avgAnnualChurnPercent * (1 - \sensitivityDelta/100)}}
\newcommand{\HCsensChurnLowPercent}{3.6}

% Theoretical lifetime with high churn
\newcommand{\lifetimeHighChurn}{\fpeval{100 / \sensChurnHighPercent}}
\newcommand{\HClifetimeHighChurn}{18.52}

% Theoretical lifetime with low churn
\newcommand{\lifetimeLowChurn}{\fpeval{100 / \sensChurnLowPercent}}
\newcommand{\HClifetimeLowChurn}{27.78}

% Capped lifetime with high churn (still hits cap)
\newcommand{\cappedLifetimeHighChurn}{\fpeval{min(\lifetimeHighChurn, \ltvCapYears)}}
\newcommand{\HCcappedLifetimeHighChurn}{7.00}

% Capped lifetime with low churn (still hits cap)
\newcommand{\cappedLifetimeLowChurn}{\fpeval{min(\lifetimeLowChurn, \ltvCapYears)}}
\newcommand{\HCcappedLifetimeLowChurn}{7.00}

% LTV with high churn (uses capped lifetime)
\newcommand{\ltvHighChurn}{\fpeval{\subAnnualGrossProfit * \cappedLifetimeHighChurn}}
\newcommand{\HCltvHighChurn}{547.62}

% LTV with low churn (uses capped lifetime)
\newcommand{\ltvLowChurn}{\fpeval{\subAnnualGrossProfit * \cappedLifetimeLowChurn}}
\newcommand{\HCltvLowChurn}{547.62}

% LTV:CAC with high churn
\newcommand{\ltvCacHighChurn}{\fpeval{round(\ltvHighChurn / \cacDigital, 0)}}
\newcommand{\HCltvCacHighChurn}{26}

% LTV:CAC with low churn
\newcommand{\ltvCacLowChurn}{\fpeval{round(\ltvLowChurn / \cacDigital, 0)}}
\newcommand{\HCltvCacLowChurn}{26}

% ============================================================================
% SCENARIO C: GROSS MARGIN SENSITIVITY
% ============================================================================

% Gross margin +20% scenario (capped at 100%)
\newcommand{\sensGrossMarginHigh}{\fpeval{min(\subGrossMargin * (1 + \sensitivityDelta/100), 100)}}
\newcommand{\HCsensGrossMarginHigh}{79.97}

% Gross margin -20% scenario
\newcommand{\sensGrossMarginLow}{\fpeval{\subGrossMargin * (1 - \sensitivityDelta/100)}}
\newcommand{\HCsensGrossMarginLow}{53.31}

% Annual GP/sub with high margin
\newcommand{\annualGPHighMargin}{\fpeval{\markWeightedAvgAnnual * \sensGrossMarginHigh/100}}
\newcommand{\HCannualGPHighMargin}{93.88}

% Annual GP/sub with low margin
\newcommand{\annualGPLowMargin}{\fpeval{\markWeightedAvgAnnual * \sensGrossMarginLow/100}}
\newcommand{\HCannualGPLowMargin}{62.59}

% LTV with high margin
\newcommand{\ltvHighMargin}{\fpeval{\annualGPHighMargin * \ltvYearsUsed}}
\newcommand{\HCltvHighMargin}{657.14}

% LTV with low margin
\newcommand{\ltvLowMargin}{\fpeval{\annualGPLowMargin * \ltvYearsUsed}}
\newcommand{\HCltvLowMargin}{438.10}

% LTV:CAC with high margin
\newcommand{\ltvCacHighMargin}{\fpeval{round(\ltvHighMargin / \cacDigital, 0)}}
\newcommand{\HCltvCacHighMargin}{31}

% LTV:CAC with low margin
\newcommand{\ltvCacLowMargin}{\fpeval{round(\ltvLowMargin / \cacDigital, 0)}}
\newcommand{\HCltvCacLowMargin}{21}

% Payback with high margin
\newcommand{\paybackHighMargin}{\fpeval{round(\cacDigital / (\markWeightedAvgMonthly * \sensGrossMarginHigh/100), 1)}}
\newcommand{\HCpaybackHighMargin}{2.7}

% Payback with low margin
\newcommand{\paybackLowMargin}{\fpeval{round(\cacDigital / (\markWeightedAvgMonthly * \sensGrossMarginLow/100), 1)}}
\newcommand{\HCpaybackLowMargin}{4.0}

% ============================================================================
% SCENARIO D: COMBINED STRESS TEST (Pessimistic)
% ============================================================================
% CAC +20%, Churn +20%, Margin -20% simultaneously

\newcommand{\stressLTV}{\fpeval{\annualGPLowMargin * \cappedLifetimeHighChurn}}
\newcommand{\HCstressLTV}{438.10}

\newcommand{\stressLtvCac}{\fpeval{round(\stressLTV / \sensCAChigh, 0)}}
\newcommand{\HCstressLtvCac}{17}

\newcommand{\stressPayback}{\fpeval{round(\sensCAChigh / (\markWeightedAvgMonthly * \sensGrossMarginLow/100), 1)}}
\newcommand{\HCstressPayback}{4.8}

% ============================================================================
% SCENARIO E: COMBINED UPSIDE (Optimistic)
% ============================================================================
% CAC -20%, Churn -20%, Margin +20% simultaneously

\newcommand{\upsideLTV}{\fpeval{\annualGPHighMargin * \cappedLifetimeLowChurn}}
\newcommand{\HCupsideLTV}{657.14}

\newcommand{\upsideLtvCac}{\fpeval{round(\upsideLTV / \sensCAClow, 0)}}
\newcommand{\HCupsideLtvCac}{39}

\newcommand{\upsidePayback}{\fpeval{round(\sensCAClow / (\markWeightedAvgMonthly * \sensGrossMarginHigh/100), 1)}}
\newcommand{\HCupsidePayback}{2.1}

% ============================================================================
% PERCENTAGE CHANGES VS BASE CASE (for display)
% ============================================================================

\newcommand{\pctChangeHighCAC}{\fpeval{round((\ltvCacHighCAC - \ltvCacRatio) / \ltvCacRatio * 100, 0)}}
\newcommand{\pctChangeLowCAC}{\fpeval{round((\ltvCacLowCAC - \ltvCacRatio) / \ltvCacRatio * 100, 0)}}
\newcommand{\pctChangeHighMargin}{\fpeval{round((\ltvCacHighMargin - \ltvCacRatio) / \ltvCacRatio * 100, 0)}}
\newcommand{\pctChangeLowMargin}{\fpeval{round((\ltvCacLowMargin - \ltvCacRatio) / \ltvCacRatio * 100, 0)}}
\newcommand{\pctChangeStress}{\fpeval{round((\stressLtvCac - \ltvCacRatio) / \ltvCacRatio * 100, 0)}}
\newcommand{\pctChangeUpside}{\fpeval{round((\upsideLtvCac - \ltvCacRatio) / \ltvCacRatio * 100, 0)}}

% ============================================================================
% END OF SOFTCODED VARIABLES
% ============================================================================
